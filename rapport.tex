\documentclass[a4paper,10pt]{article}
\usepackage[utf8]{inputenc}
\usepackage{graphicx}
\usepackage[a4paper]{geometry} %pour les marges
\geometry{hscale=0.77,vscale=0.85,centering}

\title{Configuration de routeur Cisco}
\author{Groupe 1\\Alicia \textsc{Galina}, Corentin \textsc{Ducruet}, Jason \textsc{Bury}, Ihsène \textsc{Jemaiel}}
\date{3 octobre 2016}
\begin{document}
\maketitle

\section{Introduction}
En suivant une certaine topologie (figure: \ref{fig:topo}),
nous devions d'abord configurer le routeur qui nous était attribué pour que nous puissions envoyer des pings aux PC des autres membres de notre groupe,
c'est-à-dire aux PC faisant partie de notre VLAN.
Ensuite, il fallait configurer notre routeur pour pouvoir envoyer des pings aux PC faisant partie d'autres VLAN.
Notre groupe est le groupe 1.

\section{Topologie}
\begin{figure}[!h]
 \centering
 \includegraphics[height=7cm]{topo.jpg}
 \caption{Topologie à suivre pour ce travail pratique}
 \label{fig:topo}
\end{figure}%
Le réseau est composé de 9 VLAN ainsi que de 9 routeurs qui sont connectés les uns aux autres (figure: \ref{fig:topo}). Nous avons le routeur R1 (Cisco modèle 2600) qui est connecté au routeur R4.

\section{Configuration}
\subsection{Se connecter}
Une fois connecté au \textit{Secure Console Server} du routeur R1 via l'Access Point sans-fil, nous accédons à l'invite de commande du routeur.
La commande suivante nous permet de passer en mode privilégié:
\begin{verbatim}
R1>enable
\end{verbatim}
Notre routeur a deux interfaces:
\begin{itemize}
 \item \textbf{Interface 1/0:} Faisant partie du sous-réseau contenant notre VLAN.
 \item \textbf{Interface 0/0:} Faisant partie du sous-réseau contenant R1 et R4 et la connexion Point-to-point entre eux.
\end{itemize}

\subsection{Configurer l'interface 1/0}
L'étape suivante consiste à configurer l'interface 1/0.
Les IP des réseaux attribués aux groupes sont de la forme 192.168.10$X$.0 où $X$ est le numéro de groupe.
Nous avons choisis d'attribuer à cet interface la plage d'IP correspondant à 192.168.101.0 avec le masque 255.255.255.240.

\textbf{Jusification du masque:}Nous disposons d'un switch à 4 ports donc nous ne pouvons pas connecter directement plus de 3 PC au routeur.
Nous avons donc besoin d'attribuer une plage d'adresses IP de sous-réseaux à 4 noeuds.
La première (réseau) et la dernière (broadcast) adresses étant réservées, la plage d'IP doit pouvoir contenir au moins 6 adresses et donc le nombre de bits minimum à réserver pour les adresses de sous-réseau est de 3 pour avoir une plage de 8 adresses.
Ce qui correspond à un masque de /29.
Mais, comme ce masque ne semble pas être accepté par le routeur Cisco, nous avons mis un masque de /28 qui est accepté.
Pour obtenir le masque dans la notation complète, on utilise la notation d'une adresse IP où tous les bits sont à 1 sauf les bits à réserver pour l'adressage du sous-réseau, qui doivent être les bits de poids le plus faible.
\begin{verbatim}
R1#conf t
R1(config)#int Ethernet1/0
R1(config-if)#ip address 192.168.101.1 255.255.255.240
R1(config-if)#no shutdown
R1(config-if)#exit
R1(config)#exit
R1#sh int
\end{verbatim}
Depuis un PC connecté manuellement à notre VLAN, un ping a été envoyé au routeur avec succès.

\subsection{Service DHCP}
Ensuite, nous avons tenté de configurer un serveur DHCP dans le routeur pour notre VLAN.
Remarquez dans notre historique de commande que nous avons exclu l'adresse 192.168.101.1 pour ne pas qu'elle soit attribuée automatiquement à un de nos PC.
En effet, nous voulons réserver cette adresse pour l'interface du routeur de notre VLAN.
\begin{verbatim}
R1(config)#ip dhcp excluded-address 192.168.101.1 255.255.255.255
R1(config)#ip dhcp pool g1dhcp
R1(dhcp-config)#network 192.168.101.0 255.255.255.240
R1(dhcp-config)#dns-server 192.168.101.1
R1(dhcp-config)#default-router 192.168.101.1
R1(dhcp-config)#lease 1 0 0
R1(dhcp-config)#exit
R1(config)#service dhcp
R1(config)#exit
R1#show ip dhcp server statistics
\end{verbatim}
Le service ne semblait pas opérationnel et la commande affichant les informations sur les adresses IP qui peuvent être attribuées automatiquement ne fonctionnait pas ou était différente des documentations que nous avions consulté.

\subsection{Service DNS}
Passons à la configuration du service DNS.
\begin{verbatim}
R1#conf t
R1(config)#ip domain name groupe1.com
R1(config)#ip domain-lookup
R1(config)#ip name-server 4.2.2.5
R1(config)#ip host Corentin 192.168.101.3
R1(config)#ip dns server
\end{verbatim}
La dernière commande n'était pas reconnue. Il n'était donc pas possible d'activer le DNS.

\subsection{Configurer l'interface 0/0}
Cette fois, avec le groupe 4, nous nous sommes mis d'accord pour utiliser la plage d'adresse 10.11.11.0 avec le masque 255.255.255.248.

Pour le masque, puisque nous avons juste 2 routeurs dans ce sous-réseau, il nous faut 2 adresses IP à attribuer en plus des deux adresses qui seront d'office réservées. Le masque /30 n'étant pas accepté par le routeur, nous avons mis un masque /29.
L'adresse 10.11.11.0 et 10.11.11.7 seront donc réservées et nous avons décidé que le routeur R4 aurait l'adresse 10.11.11.1 et le routeur R1 l'adresse 10.11.11.2.
\begin{verbatim}
R1#conf t
R1(config)#int Ethernet0/0
R1(config-if)#ip address 10.11.11.2 255.255.255.248
R1(config-if)#no shutdown
R1(config)#exit
R1#sh int
\end{verbatim}
Depuis un PC de connecté à notre VLAN, nous avons pu envoyer un ping au routeur R4.

\section{Service OSPF}
Afin d'obtenir les adresses IP des machines accessibles et connectés au routeur R4, nous avons utilisé le service OSPF.
Dès que nous entrons la commande \texttt{router ospf 1}, le routeur doit recevoir des messages de protocoles OSPF venant du routeur R4.
Le groupe 4 nous avait signalé que le service OSPF du routeur R4 était configuré. Nous avons tenté de le configurer à notre tour.
\begin{verbatim}
R1(config)#router ospf 1
R1(config-router)#network 192.168.101.0 0.0.0.15 area 1
R1(config-router)#exit
R1(config)#exit
R1#sh ip route
\end{verbatim}
Nous observons qu'aucune nouvelle adresse IP n'a été découverte via le service OSPF.

\end{document}
